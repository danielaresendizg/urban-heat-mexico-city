\section{Appendix A. Climatic Data Processing in Google Earth Engine}
\label{app:a}

This study implemented a custom script in Google Earth Engine (GEE) to process Landsat 8/9 Level-2 surface reflectance imagery and produce a decadal (2014--2024) climatology of surface and air temperatures for Mexico City. The workflow proceeded as follows:

\subsection*{Area of Interest and Parameters}
Defined as a bounding polygon for Mexico City; spatial resolution set to 30 m. RedMet-derived regression coefficients were applied for temperature calibration.

\subsection*{Preprocessing}
Optical bands scaled to reflectance; thermal bands to brightness temperature. Cloud, shadow, and cirrus pixels masked using QA\_PIXEL bits.

\subsection*{Summer Composites}
Median images computed for June--August of each year. A multi-annual average generated the decadal climatology.

\subsection*{Spectral Indices}

The Normalised Difference Vegetation Index (NDVI) was calculated as:
\begin{equation}
\text{NDVI} = \frac{\rho_{NIR} - \rho_{Red}}{\rho_{NIR} + \rho_{Red}}
\label{eq:ndvi}
\end{equation}

The Normalised Difference Built-up Index (NDBI) was calculated as:
\begin{equation}
\text{NDBI} = \frac{\rho_{SWIR} - \rho_{NIR}}{\rho_{SWIR} + \rho_{NIR}}
\label{eq:ndbi}
\end{equation}

Broadband albedo ($\alpha$) was estimated via mean reflectance across shortwave bands:
\begin{equation}
\alpha = \frac{1}{n} \sum_{i=1}^{n} \rho_i
\label{eq:albedo}
\end{equation}

\subsection*{Emissivity and LST}

Surface emissivity ($\varepsilon$) was estimated from NDVI-based vegetation fraction ($P_v$):
\begin{equation}
P_v = \left( \frac{\text{NDVI} - \text{NDVI}_{min}}{\text{NDVI}_{max} - \text{NDVI}_{min}} \right)^2
\label{eq:pv}
\end{equation}

\begin{equation}
\varepsilon = 0.004 \cdot P_v + 0.986
\label{eq:emissivity}
\end{equation}

Land Surface Temperature (LST) was calculated using the single-channel radiative transfer model:
\begin{equation}
\text{LST} = \frac{T_B}{1 + \left( \frac{\lambda \cdot T_B}{\rho} \right) \ln(\varepsilon)}
\label{eq:lst}
\end{equation}

where $T_B$ is brightness temperature, $\lambda$ is the wavelength of emitted radiance, and $\rho = h \cdot c / \sigma$ (with $h$ = Planck's constant, $c$ = speed of light, $\sigma$ = Boltzmann constant).

\subsection*{Air Temperature Calibration}

Air temperature ($T_a$) was derived from LST using empirical calibration with RedMet observations:
\begin{equation}
T_a = 0.5540 \times \text{LST} + 5.7606
\label{eq:ta_calibration}
\end{equation}

\subsection*{Derived Thermal Metrics}

Urban Heat Island intensity (UHI\_air) was computed as the z-score of calibrated air temperature:
\begin{equation}
\text{UHI}_{air} = \frac{T_a - \bar{T}_a}{\sigma_{T_a}}
\label{eq:uhi}
\end{equation}

The Urban Thermal Field Variance Index (UTFVI\_air) was also derived following Waleed et al. (2023).

\subsection*{Export}
Final rasters---including NDVI, NDBI, albedo, LST, $T_a$, UHI\_air, and UTFVI\_air---were exported to Google Drive at 30 m resolution.
