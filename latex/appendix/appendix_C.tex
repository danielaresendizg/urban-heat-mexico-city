\section{Appendix C. Social Indicators}
\label{app:c}

Socio-demographic indicators were obtained from the 2020 Mexican Population and Housing Census (INEGI, 2021) at the urban-block (manzana) level. The following table presents the complete list of indicators used in this study, organized by thematic group.

\begin{footnotesize}
\begin{longtable}{@{}llll@{}}
\caption{Social indicators based on INEGI's manual.} \label{tab:social_indicators} \\
\toprule
\textbf{Group} & \textbf{Indicator} & \textbf{Variable} & \textbf{Description} \\
\midrule
\endfirsthead

\multicolumn{4}{c}%
{{\tablename\ \thetable{} -- continued from previous page}} \\
\toprule
\textbf{Group} & \textbf{Indicator} & \textbf{Variable} & \textbf{Description} \\
\midrule
\endhead

\midrule
\multicolumn{4}{r}{{Continued on next page}} \\
\endfoot

\bottomrule
\endlastfoot

Age & pct\_0a5 & P\_0A2 + P\_3A5 & People aged 0--5 \\
Age & pct\_6a14 & P\_6A11 + P\_8A14 & People aged 6--14 \\
Age & pct\_15a59 & POB15\_64 & Population 15--64 \\
Age & pct\_65plus & POB65\_MAS & Population 65+ \\
\midrule
Ethnicity & pct\_ethnic\_afro & POB\_AFRO & Afro-descendant (3+) \\
Ethnicity & pct\_ethnic\_ind & P3YM\_HLI & Indigenous speakers (3+) \\
Ethnicity & pct\_ethnic\_other & Derived & Other ethnicity \\
\midrule
Disability & pct\_with\_disc & PCON\_DISC & With disability \\
Disability & pct\_without\_disc & Derived & Without disability \\
\midrule
Education & pct\_no\_school & P15YM\_SE & No schooling (15+) \\
Education & pct\_elementary\_edu & P15PRI\_CO & Elementary (15+) \\
Education & pct\_elementary2\_edu & P15SEC\_CO & Secondary (15+) \\
Education & pct\_more\_edu & P18YM\_PB & Higher education (18+) \\
\midrule
Economy & pct\_ocup & POCUPADA & Employed (12+) \\
Economy & pct\_desocup & PDESOCUP & Unemployed (12+) \\
Economy & pct\_inac & PE\_INAC & Inactive (12+) \\
\midrule
Health & pct\_serv\_med & PDER\_SS & With health service \\
Health & pct\_no\_serv\_med & PSINDER & Without health service \\
\midrule
Mobility & pct\_pop\_car & Derived & With car access \\
Mobility & pct\_pop\_without\_car & Derived & Without car access \\
\midrule
Care & rel\_dep\_0\_14 & Derived & Dependency ratio (0--14) \\
\midrule
Gender & rel\_h\_m & POBMAS/POBFEM & Male-to-female ratio \\

\end{longtable}
\end{footnotesize}

These indicators were selected based on their relevance to thermal vulnerability:
\begin{itemize}
    \item \textbf{Physiological sensitivity}: Age groups (children, elderly) and disability status.
    \item \textbf{Adaptive capacity}: Access to health services, education level, economic activity.
    \item \textbf{Exposure factors}: Mobility indicators (car access).
\end{itemize}
