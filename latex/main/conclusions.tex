\section{Conclusions}

 5. Discussion

This dissertation set out to move from surface proxies to lived exposure and from citywide averages to block- and corridor-scale analysis in Mexico City. Three findings stand out. First, Tmrt---rather than LST---governs pedestrian heat stress at diurnal time under the city’s high-elevation, dry-air conditions. This is consistent with urban biometeorology: SOLWEIG/UMEP resolves short- and long-wave fluxes and shows that Tmrt is the main radiative load perceived by pedestrians in complex street canyons(Lindberg, Holmer and Thorsson, 2008; Lindberg and Grimmond, 2011). In our cases, blocks with air temperature (Ta) in the 26--28 $^{\circ}$C band still produced Tmrt values above 60 $^{\circ}$C and PET above 41 $^{\circ}$C, demonstrating why LST or Ta alone understate street-level risk and why design targets should be formulated against biometeorological rather than purely surface metrics.

Second, exposure is unequal and socially structured. The macro analysis shows strong co-location between hotter blocks ($\geq$ 26 $^{\circ}$C; $\geq$ 28 $^{\circ}$C) and socially sensitive groups such as older adults, people with disabilities and carless households, with borough-heterogeneous associations. This pattern mirrors international evidence of systematic disparities in urban heat exposure by social vulnerabilities. Global health monitoring further reports steep rises in heat-related mortality and labour losses, which disproportionately affect disadvantaged groups (Romanello et al., 2024; WHO, 2024).

Third, two reinforcing mechanisms translate morphology and networks into lived heat. Structural heat arises in compact, sealed fabrics: higher GSI correlates with lower NDVI and higher Ta, while typologies such as dense perimeter blocks and continuous compact forms concentrate risk, in line with Space Matrix theory on how intensity and coverage modulate sky-view and evapotranspiration. Corridor heat emerges where space syntax centralities peak: highly integrated or through-movement corridors systematically coincide with hotter segments, explaining why least-angle pedestrian spines often combine flows and exposure.

These findings answer the research questions directly: the co-location of heat and social sensitivity under locally relevant thresholds confirms that risk is geographically patterned; hotspot clusters are best characterised by form--network couplings; and micro-scale simulations show that radiative peaks concentrate on highly integrated corridors, converting remote-sensing signals into actionable street segments.

The policy implication is that heat adaptation should be reframed around measurable outcomes. A Key Performance Indicator (KPI) is a quantifiable metric linked to a specific objective\citep{Parmenter2015}. In this context, generic outputs such as the number of trees planted are weak substitutes for outcome-oriented KPIs that track cooling delivered where people walk and dwell. Suitable KPIs for Mexico City could include (for example) median midday Tmrt reductions of 10--15 $^{\circ}$C along priority corridors, continuous shade over at least 70\% of pedestrian length between 10:00 and 14:00, the provision of cooling refuges every 300--500 m, and the reduction of hot-segment length (Ta $\geq$ 28 $^{\circ}$C at p90). These indicators can be derived from local and low-cost data such as Landsat imagery, open building footprints, census statistics and simple microclimate modelling, and can support learn-as-you-go governance with feedback loops.

The strategic rationale is twofold. First, passive cooling---through shade, ventilation and non-glare cool materials---reduces cooling demand, avoiding dependence on energy-intensive air conditioning and mitigating the risk of energy poverty under warming. International assessments emphasise that halving emissions from cooling depends on passive design, efficiency and refrigerant transition, with passive measures indispensable because they reduce need before efficiency improves use (UNEP, 2021). In Latin America, where energy poverty is a pressing concern, heat-resilient public space can directly lower the energy burden of low-income households\citep{Gonzalez2023}. Second, maladaptation risks are real. Reflective pavements may reduce surface temperatures but increase Tmrt for pedestrians at midday if implemented without shading. The Phoenix Cool Pavement pilot documented higher noontime Tmrt on high-albedo coatings compared to asphalt \citep{CityPhoenix2024}. The \citet{IPCC2022} similarly warns that poorly specified measures can shift or even amplify risks and calls for robust monitoring--evaluation frameworks.

Methodologically, this study demonstrates a city-to-street pipeline---calibration of satellite and ground data, spatially explicit social sensitivity, coupling of form and network, and micro-scale biometeorology---producing decision-ready KPIs for corridor-based interventions. This bridges the gap between big-data screening and site-specific design and resonates with programmes that move from lists of actions to performance-tracked delivery. Limitations remain, including calibration focused on midday hours, uncertainties in emissivity estimates from NDVI, topology affecting syntax metrics, simplified UMEP assumptions excluding anthropogenic heat, and socio-demographic variables acting as proxies rather than direct measures of exposure. Yet triangulation across scales produced consistent patterns with clear operational implications.

This thesis began with an urgent question: how can we move from observing urban heat from above---through satellite maps---to understanding it as a lived experience in Mexico City’s streets? Addressing this challenge required a multi-scalar framework. A decadal summer climatology, calibrated with Landsat 8/9 and RedMet observations, was integrated with configurational, morphological and census-based metrics to identify thermo-social hotspots: areas where thermal intensity overlaps with social vulnerability. This represents a shift from surface cartography to operational frameworks capable of guiding targeted urban interventions.

Three findings stand out. First, under conditions of high irradiance and dry air, mean radiant temperature (Tmrt)---rather than land surface temperature (LST) or air temperature (Ta)---governs pedestrian heat stress at midday, underscoring the need to anchor design targets in biometeorological rather than purely surface metrics. Second, by combining thermal layers with Space Syntax, Space Matrix and socio-demographic predictors, exposure becomes concrete: high-flow corridors lacking shade and morphological types such as dense perimeter and compact low-rise emerge as areas where heat and disadvantage converge. Third, the thermo-social hotspot approach ensures that priorities for adaptation are located precisely where thermal risk and vulnerability coincide in statistically significant ways.

The contribution is twofold. Substantively, the research demonstrates that thermal justice is a spatial reality: the corridors and neighbourhoods where heat and social disadvantage converge follow urban and social structures, not random distributions. Methodologically, it advances a replicable bridge linking metropolitan screening to micro-climatic street-scale simulation using open tools, enabling multi-level interventions that integrate science and design practice.

This framework also connects to a global challenge. Rising planetary temperatures are driving soaring demand for cooling: the International Energy Agency projects that, without efficiency measures, electricity use for space cooling will triple by 2050, with steep demand peaks in emerging economies \citep{IEA2018}. At the same time, the growing reliance on air conditioners and fans creates an urban vicious circle: extreme heat accelerates device use, whose condensers expel residual heat into the street, raising nocturnal air temperatures in dense districts by up to 1--2 $^{\circ}$C and reinforcing demand the following day (Sailor, 2014; Salamanca et al., 2014; Santamouris, 2016).

In this context, passive solutions acquire strategic value. While the transition to cleaner and more efficient technologies remains indispensable, shade, vegetation and cool materials are irreplaceable in public space. Designing streets that lower Tmrt, provide continuous shade and ensure accessible refuges simultaneously protects health, relieves pressure on electricity grids and mitigates energy poverty.

In sum, this thesis goes beyond mapping heat: it translates urban science into actionable goals that cool streets, ease network loads and protect those least able to afford mechanical cooling. It closes the circle by showing that the future of urban adaptation lies in solutions that cool without warming the city further, and opens a programme of research-action to cool, equitably and sustainably, the streets that need it most in Mexico City.

Berghauser Pont, M. and Haupt, P. (2020) Spacematrix: Space, Density and Urban Form. 2nd ed. Rotterdam. Available at: https://doi.org/https://doi.org/10.59490/mg.38.

Bröde, P., Fiala, D. and others (2012) ‘Derivation and validation of the Universal Thermal Climate Index (UTCI)’, International Journal of Biometeorology, 56, pp. 481--494. Available at: https://doi.org/10.1007/s00484-011-0454-1.

Chakraborty, T. et al. (2022) ‘Lower urban humidity moderates outdoor heat stress’, AGU Advances, 3(4), p. e2022AV000729. Available at: https://doi.org/10.1029/2022AV000729.

City of Phoenix Street Transportation Department (2024) Cool Pavement Program -- Phase II Report. Available at: https://www.phoenix.gov/content/dam/phoenix/streetssite/documents/cool-pavement/CoolPavement\_Phase2\_Report\_FINALOct24.pdf.

ETH/ Singapour (2017) ‘Strategies for Cooling Singapore: A Catalogue of Urban Heat Mitigation Measures’. Singapore. Available at: https://doi.org/https://doi.org/10.3929/ethz-b-000258216.

