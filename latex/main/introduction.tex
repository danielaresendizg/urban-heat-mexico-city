\section{Introduction}

Cities are experiencing a rapid escalation of heat-related risks with measurable health, economic and urban consequences worldwide. In 2024, the planet recorded the warmest year since instrumental observations began, with concurrent records in mean temperature, ocean heat content and cryospheric change(NASA, 2025; NOAA, 2025; Rohde, 2025; WHO, 2025). Epidemiological and macroeconomic evidence underscores the scale of the challenge: the Lancet Countdown 2024 reports record highs in heat-attributable mortality among those aged 65 and over and in productivity losses due to thermal exposure \citep{Romanello2024}, while labour projections from the \citet{ILO2019} anticipate global losses equivalent to tens of millions of jobs from heat stress. The Intergovernmental Panel on Climate Change (IPCC) emphasises that cities concentrate exposure, vulnerabilities and agency, demanding explicitly urban policies and metrics (2022). The World Health Organization (WHO) likewise stresses the rising burden of heat mortality and the urgency of robust, up-to-date heat--health action plans (2025).

Within this context, Mexico City is a critical case where intense urban heat intersects with social and spatial inequalities that amplify vulnerability. Three practical gaps motivate this thesis: identifying where exposure is concentrated and who vulnerable groups are the most affected; explaining which features of urban form and street network sustain these hotspots; and translating this understanding into verifiable passive design targets. Addressing these gaps requires a move from broad diagnosis to operational delivery.

In the context of this study, local daytime summer air-temperature thresholds of 26 $^{\circ}$C or more at block scale are relevant for risk communication in Mexico City (Magaña and Vargas, 2020; World Meteorological Organization, 2025b). Moreover, outdoors, pedestrian load is governed primarily by mean radiant temperature (Tmrt), motivating the use of biometeorological indices such as the Universal Thermal Climate Index (UTCI) and Physiological Equivalent Temperature (PET) (Höppe, 1999; Lindberg, Holmer and Thorsson, 2008; Bröde, Fiala and others, 2012; Jendritzky, de Dear and Havenith, 2012). Spatial configuration is measured with Space Syntax indicators of Integration and Choice, normalised as Normalised Angular Integration (NAIN) and Normalised Angular Choice (NACH), while morphology is described with Space Matrix metrics---Floor Space Index (FSI), Ground Space Index (GSI), Linear Density (L) and Open Space Ratio (OSR)---linking accessibility and microclimate (Hillier and Hanson, 1984; Hillier, Yang and Turner, 2012b; Berghauser Pont and Haupt, 2020).

The thesis is structured around three research questions:

Q1. Where, and to what extent, do daytime summer mean air temperatures at block level co-locate with socially and physiologically vulnerable groups in Mexico City, and how heterogeneous across municipalities are these associations under thresholds of 26 $^{\circ}$C or more?

Q2. Which combinations of urban form/density and network configuration best characterise thermal--social hotspots, and which contiguous clusters jointly maximise thermal severity and pedestrian accessibility for passive-cooling simulation?

Q3. In highly integrated, strongly exposed pedestrian corridors, how can micro-scale thermal stress---together with geometric parameters---be assessed to define passive design targets capable of reducing heat?

These questions form a logical chain: Q1 establishes where and with whom to intervene; Q2 clarifies why, form and network, and identifies actionable clusters; Q3 specifies how much and with what to reduce space heat via verifiable targets.

Methodologically, the thesis proposes a city-to-street pipeline: (i) calibrating satellite thermal surfaces with local observations to estimate daytime air temperature (Ta); (ii) overlaying thermal exposure (Ta) with social sensitivity to identify conservative hotspots; (iii) characterising hotspots with Space Matrix and Space Syntax to locate contiguous clusters combining thermal severity and high pedestrian accessibility; and (iv) simulating biometeorological indices at micro scale in highly integrated corridors to define passive design targets (continuous shade, non-glare cool materials, ventilation). This approach responds to IPCC/WHO calls to shift from generic action lists to performance-oriented delivery with iterative learning and equity. It operationalises Q1 as a robust block-level thermal--social diagnosis, Q2 as cluster selection based on form plus network, and Q3 as the translation of biometeorological simulations into quantifiable objectives for urban design and management.

The study focuses on Mexico City, using the block as the unit of analysis with data from the National Institute of Statistics and Geography INEGI (2021) and the summer period 2014--2024 (1 June--30 August). Ta is calibrated from Land Surface Temperature (LST) using observations from the Mexico City Automatic Weather Station Network (SEDEMA-CDMX, 10:00--14:00), also informing microclimatic forcing, and air-temperature Urban Heat Island (UHI) maps are derived in Google Earth Engine. Morphology comes from Open Buildings from Google and cadastral data (Instituto de Planeación Democrática y Prospectiva CDMX), and network configuration from a segmented graph with NAIN/NACH at 500, 1000, 1500 and 5000 m. Limitations include emissivity and humidity uncertainty, exclusion of anthropogenic heat in the Solar and LongWave Environmental Irradiance Geometry model (SOLWEIG, part of the Urban Multi-scale Environmental Predictor, UMEP), and potential topological biases; triangulation across scales (macro--meso--micro) and sources nevertheless yields a consistent picture \citep{Lindberg2008}.

Chapter 2 consolidates the conceptual framework of precise definitions, the methodological shift from surface temperature to lived exposure via calibrated LST, Tmrt, UTCI and PET supported by SOLWEIG/UMEP (Lindberg et al., 2008; Höppe, 1999; Jendritzky et al., 2012; Bröde et al., 2012), and evidence on thermal inequality(Jendritzky, de Dear and Havenith, 2012; Hoffman, Shandas and Pendleton, 2020). It presents Space Syntax and Space Matrix theories as the twin spatial lenses (Hillier and Hanson, 1984; Hillier, Yang and Turner, 2012a; Berghauser Pont and Haupt, 2020), and situates Mexico City’s basin dynamics, gap winds and ventilation corridors(de Foy, Fast, Silke J Paech, et al., 2006b, 2006a; de Foy, Fast, Stephen J Paech, et al., 2006), classical UHI evidence , and the case for locally relevant thresholds \citep{MaganaVargas2020}. Policy exemplars---Cooling Singapore and Bangkok’s plan---are contrasted with local frameworks, such as, the Local Climate Action Strategy(ELAC 2021--2050) and the Climate Action Programme of Mexico City (PACCM 2021--2030), which still lack local level performance indicators using Ta or Tmrt \citep{SEDEMA2021}.

Chapter 3 describes the methods across three scales. At the macro level, Landsat 8/9 imagery (2014--2024) was processed with radiometric corrections, NDVI, NDBI and shortwave albedo, while Ta was calibrated against RedMet records to derive UHI\_air in Earth Engine. Socio-demographic data from INEGI represented vulnerable groups, and urban morphology and street-network metrics provided structural indicators. GWR by municipality identified thermal--social hotspots using 26/28 $^{\circ}$C thresholds. At the meso level, segment statistics and the share of street length above these thresholds revealed contiguous clusters linking heat severity with accessibility. At the micro level, UMEP/SOLWEIG simulations based on high-resolution DEM/DSM, Open Buildings and multi-annual NDVI produced Tmrt fields, from which UTCI and PET were calculated for 10:00, 12:00 and 14:00 along highly integrated corridors.

Chapter 4 presents the results. At the macro scale, the 2014--2024 summer climatology produced Ta and hotspot maps (26/28 $^{\circ}$C), while GWR revealed strong spatial heterogeneity. At the meso scale, hotter zones (hazard 2) proved more integrated and less vegetated, with compact and perimeter typologies (high GSI, low OSR) concentrating risk. At the micro scale, simulations for three cases---the General Hospital, the Airport and the Basilica of Guadalupe---showed that diurnal Tmrt values exceeding 60 $^{\circ}$C and UTCI/PET in strong-stress ranges, closely aligned with highly integrated corridors.

Chapter 5 draws the conclusions and policy implications. It highlights three main contributions (1) Tmrt as the most appropriate indicator of pedestrian heat stress in Mexico City; (2) the clear overlap of thermal exposure and social vulnerability at sub-municipal scales; (3) and two distinct spatial mechanisms---structural heat in compact, low-vegetation areas and corridor heat along highly integrated axes. The chapter calls for performance-based KPIs to guide resilient governance, cautions against maladaptive solutions such as reflective pavements without shade, and emphasises passive strategies---continuous shade, ventilation and cool non-glare materials---to reduce cooling demand and avoid energy poverty. These recommendations are tied to ELAC and PACCM through measurable local level targets.

