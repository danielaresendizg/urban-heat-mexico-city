\section{Literature Review}

Urban heat is no longer a niche climatological curiosity; it is a rapidly escalating public-health, economic and planning concern. Record-breaking global conditions and compounding extremes in recent years have intensified exposure and impacts across regions and income levels, with disproportionate burdens on vulnerable populations (Romanello et al., 2024; World Meteorological Organization, 2025b). A first conceptual clarification is essential for this work: the UHI describes urban--rural thermal contrasts arising from materials and form, whereas extreme heat refers to absolute hazardous conditions and heatwaves at a given location. They interact---UHIs can amplify heatwaves---but they are not the same, and conflating them weakens risk management and design responses (Stewart and Oke, 2012; IPCC, 2022). A second clarification concerns governance: effective heat-risk reduction depends on locally relevant evidence and on institutions capable of acting on city-scale determinants---shade, materials, vegetation, ventilation, operations and warnings---rather than relying solely on national or global metrics \citep{WHO2024}. Consistent with Latin American disaster-risk scholarship, heat risk is socially constructed through exposure, sensitivity and adaptive capacity shaped by development choices and, therefore, is actionable within municipal mandates\citep{Lavell1999,Lavell2003}. For Mexico City in particular, regional climate work emphasises the need to characterise warm spells in their local hydro-climatic context to tune urban and public-health responses\citep{MaganaVargas2020}.

At the global scale, the most dangerous combinations of heat and humidity already occur in parts of South and Southeast Asia, the Middle East and North Africa, and swathes of Africa; large Latin American cities sit in an intermediate but worsening band of risk as warming, urbanisation and sociodemographic change co-evolve (IPCC, 2022; WHO, 2024) Against this background, the literature has shifted from surface-centric proxies toward metrics that better capture lived exposure. Daytime LST can relate weakly to air temperature and human heat stress within cities, and in some contexts reductions in urban humidity can partially offset higher air temperatures, moderating stress during critical hours\citep{Chakraborty2022}. Because radiant load at pedestrian height governs much of the outdoor thermal experience, risk assessment and intervention appraisal increasingly rely on air-based and biometeorological metrics (e.g., Tmrt and composite indices such as UTCI) rather than on LST maps alone. Modelling tools embedded in city-scale workflows---such as SOLWEIG within UMEP for QGIS---allow planners to estimate spatial patterns of Tmrt under different shading and material scenarios, supporting analysis that is broad enough for policy while sensitive to street conditions \citep{Lindberg2008}.

A parallel and equally important strand foregrounds equity. The IPCC’s framing of vulnerability---as the intersection of exposure, sensitivity and adaptive capacity---has catalysed a wave of empirical studies documenting unequal burdens across age, income and ethnicity. In the United States, historically redlined neighbourhoods are significantly hotter today\citep{Hoffman2020}, and nationwide analyses find systematic disparities in UHI exposure by race and income \citep{Hsu2021}. Comparable evidence from Europe shows similar patterns: in Greater London, socio-demographic inequalities map closely onto UHI intensity, with children, ethnic minorities and lower-income groups disproportionately exposed \citep{KrenzAmann2025}. Global health monitoring echoes these findings, reporting steep rises in heat-related mortality and labour losses that fall most heavily on already disadvantaged groups, especially in the Global South \citep{Romanello2024}. Together, these findings argue for multidimensional indicators at fine spatial scales and for intersecting social data with physical layers of exposure to identify where---and for whom---interventions matter most\citep{IPCC2022}.

Building on those insights, recent work treats urban form and networks as thermal infrastructures that modulate exposure during movement and co-presence in public space. Space Syntax offers a theory and a set of measures, by Hillier and colleagues, that capture how spatial configuration shapes pedestrian flows; by modelling the cognitive costs of changes of direction, angular analyses identify least-angle paths that often predict movement better than pure metric distance. Core indicators such as Integration (accessibility) and Choice (through-movement) thus make the spatial variable explicit and comparable across scales, revealing corridors where flows concentrate and, where shade is absent, so does thermal burden(Hillier and Hanson, 1984; Hillier et al., 1987; Hillier, Yang and Turner, 2012b). In parallel, Space Matrix conceptualises density as a relational variable---balancing FSI, GSI, L and OSR---to position built form within a continuous morphological state space. Variations in coverage and height modulate sky-view factor and canyon proportions with consistent microclimatic effects: compact, high-coverage, low-porosity fabrics tend to store heat and suppress evapotranspiration, while more porous or height-differentiated forms promote midday shading and airflow\citep{BerghauserPont2020}. Read together, these frameworks distinguish “structural heat” (rooted in sealing and vegetation deficits) from “corridor heat” (concentrated along highly accessible pedestrian axes), and they provide an operational logic for passive, place-sensitive measures.

Mexico City’s basin dynamics make this form--network coupling especially salient. Weak synoptic forcing, gap winds and topographic channelling generate shifting convergence lines and ventilation corridors across the metropolis(de Foy, Fast, Silke J Paech, et al., 2006b; de Foy, Fast, Stephen J Paech, et al., 2006), while classic urban-climate work documents the magnitude and spatial patterns of the city’s UHI and the role of vegetation and form in mediating it \citep{Jauregui1997}. This evidence base supports an approach that couples calibrated air-based surfaces (Ta/RH) and pedestrian-scale biometeorology with configurational analysis to identify the intersection of lived exposure and movement intensity---essentially, where people actually walk, dwell and queue during hot parts of the day. In practical terms, this means prioritising continuous shade (trees, canopies, pergolas) along locally accessible spines, deploying reflective or otherwise “cool” materials where they do not increase radiant load for pedestrians, creating cooling refuges at high-flow nodes, and tailoring the mix to morphological typologies to maximise efficacy without undermining urban legibility(Hillier and Hanson, 1984; Stewart and Oke, 2012).

Policy and planning practice worldwide are beginning to catch up with the academic insights, but an evaluation gap persists. Bangkok’s heat-action initiatives, for example, articulate quantified levers---cooling shelters, green corridors, alert systems, heat mapping, code reforms and a proposed heat-resilience fund---supported by performance tracking and a clear economic and health rationale. Yet, like many plans, they under-specify how to deliver guaranteed reductions in street-level Ta and Tmrt along the very corridors people use \citep{WorldBank2025}. Singapore’s Strategies for Cooling Singapore goes further, cataloguing measures across form, materials, greening and ventilation that explicitly target the physical determinants of outdoor comfort, and providing a clearer bridge from policy to design and operations \citep{ETHSingapore2017}.

Against this benchmark, Mexico’s policy architecture is advancing but remains less operational. Nationally, the Ministry of Health’s surveillance system for extreme temperatures and the seasonal information to provide warnings and guidance but do not set design-level KPIs for reducing Ta or Tmrt \citep{MinistryHealth2025}. In Mexico City, the Local Climate Action Strategy 2021--2050 and the Climate Action Programme 2021--2030 explicitly acknowledge heat risks and call for shading and greening, and the city’s seasonal heat plans mobilise hydration points, and public information campaigns. However, measurable corridor-level cooling targets and a systematic monitoring--evaluation framework comparable to Singapore or Bangkok remain absent \citep{SEDEMA2021}. The problem is recognised, but the delivery of street-level performance is still under-specified.

The IPCC warns that such gaps risk maladaptation---measures that inadvertently increase risk, shift vulnerabilities or lock in future hazards---if trade-offs and distributive effects are not critically assessed (2022). Examples include reflective pavements that worsen radiant loads for pedestrians or expanded air-conditioning that increases outdoor waste heat and energy demand. Embedding ex-ante simulation and performance KPIs with both thermal and social dimensions offers a way forward: cities can target scarce resources more precisely, avoid unintended consequences and establish feedback loops for learning.

Finally, there is converging evidence that passive, network-sensitive measures---continuous shade, pedestrian-safe high-albedo and cool materials, ventilation-oriented street design and urban greening---can substantially reduce cooling demand and outdoor heat stress, especially when paired with building standards and minimum energy performance for the active stock(Khosla et al., 2021; United Nations Environment, 2023; Khosla, 2025). Given that emerging-economy cities will bear the largest marginal burdens of extreme heat, these strategies offer energy, cost and equity advantages while strengthening system resilience (IPCC, 2022; Romanello et al., 2024). In synthesis, the literature urges a pivot from surface proxies to lived exposure; from city-wide averages to block- and corridor-scale analysis; and from generic checklists to performance-tracked, equity-aware interventions. The gap this dissertation addresses is methodological and operational: integrating micro-scale social vulnerability with configurational (Space Syntax) and morphological (Space Matrix) analyses, and evaluating candidate passive measures against air-based and biometeorological metrics (e.g., Ta and Tmrt/UTCI) along the very pedestrian networks where exposure concentrates. In doing so, it responds to calls for locally grounded, learn-as-you-implement governance and offers a practical route to simulate before intervening, prioritise fairly and spend scarce resources where thermal risk and social return intersect most strongly(Lavell, 1999; IPCC, 2022).

