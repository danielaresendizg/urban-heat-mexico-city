\section{Methodology}

This chapter outlines the data sources and analytical approach used to assess urban heat exposure in Mexico City and its relationship to urban form and social vulnerability. The methodological strategy is structured across three nested spatial scales---macro, medium, and micro---allowing for a progressive analysis from citywide screening to street-level simulations to provide passive cooling strategies.

Level-2 Landsat 8/9 summer imagery (1 June--30 August, 2014--2024) was processed in Google Earth Engine using standard radiometric scaling, cloud and cirrus masking, and thermal calibration \citep{Gorelick2017}. Daytime land surface temperature was retrieved through a radiative-transfer approach with emissivity correction based on NDVI-derived vegetation fraction. The Normalised Difference Vegetation Index (NDVI) is defined as the difference between near-infrared and red reflectance; higher values indicate more photosynthetically active vegetation \citep{Tucker1979}. The Normalised Difference Built-up Index (NDBI) is calculated as the difference between short-wave infrared and near-infrared reflectance, with positive values typically identifying built-up or impervious surfaces\citep{Zha2003}. Short-wave broadband albedo, representing the fraction of incoming solar radiation reflected by the surface across the short-wave spectrum, was estimated from Landsat reflectances using narrow- to broadband conversions \citep{Liang2001}.

To account for the city’s high-altitude and low-humidity conditions, land surface temperature was empirically converted to air temperature using hourly RedMet observations between 10:00 and 14:00 from 18 stations. A city-specific model (Ta = 0.5540 × LST + 5.7606) was applied to all scenes, addressing known discrepancies between LST and air temperature and recognising the influence of humidity on heat stress\citep{Chakraborty2022}. Annual summer medians were averaged to produce a 2014--2024 climatology. From the calibrated air temperature fields, we derived UHI\_air (z-scores of Ta) and UTFVI\_air \citep{Waleed2023}. Outputs are provided at 30 m spatial resolution and constitute inputs across all scales of analysis (see more information in Appendix A and B).

Socio-demographic indicators were obtained from the 2020 Mexican Population and Housing Census \citep{INEGI2021} at the urban-block (manzana) level. The block is defined as the polygonal unit bounded by streets, paths, or other physical limits that aggregates residential, commercial, or vacant plots.

Selected indicators correspond to socially vulnerable groups physiologically more exposed to heat (aged 65+, children aged 6--14, people with disabilities, carless households, lack of health services, economic inactivity, and incomplete primary education). Indicators were expressed as percentages or ratios using official denominators (see table 1). This choice reflects evidence that these groups are the most exposed and least able to adapt to extreme heat (see more information in Appendix C).

Urban morphology was characterized at the block level using the Space Matrix framework. Building footprints were obtained from Google’s Open Buildings dataset and combined with cadastral records of built floor area to generate four key indicators: Floor Space Index (FSI), Ground Space Index (GSI), equivalent number of floors (L), and Open Space Ratio (OSR). Following the operational definitions of \citet{BerghauserPont2020}, “Building Intensity (FSI) refers to the built intensity of an area”, while “Coverage (GSI) indicates the compactness of the built environment” (p. 96). The “Building Height (L) is understood as the average number of storeys and can be derived from the relation between intensity and coverage”, that is, from FSI and GSI (2020, p. 97). Likewise, “Spaciousness (OSR) is described as the amount of non-built space at ground level per square metre of gross floor area” (2010, p. 97).

The basic measures underlying these indicators are also defined in the framework. The “Gross Floor Area (F) is the sum of all surfaces measured per floor along the perimeter of the partitions that enclose the building, including underground levels and pitched roof areas, but excluding loggias, balconies, uncovered walkways, roof terraces, and open fire escapes or emergency stairways” (2020, pp. 94--95). The “Built Area or Footprint (B) refers to the floor area measured at ground level along the perimeter of the dividing partitions of the building, excluding overhanging or underground built structures” (2020, p. 95).

Although Space Matrix is inherently continuous, a typological labelling was applied for cartographic interpretation, classifying blocks by thresholds in FSI, GSI, and L. This typology distinguishes open pavilions, row or attached low-rise housing, low to dense perimeter blocks, linear bars, towers in a park, super-compact high-rises, compact low-rise fabrics, and mixed or indeterminate forms (see table 2; more information in Appendix D).

The street network was modelled as a segment graph and analysed through angular centralities at multiple radii to capture connectivity patterns from the local to the metropolitan scale. Angular analysis treats changes of direction as the cost in shortest paths, a principle that has been shown to provide a stronger basis for predicting movement than metric distance. To enable meaningful comparisons across different systems, we focus on the normalised measures of integration and choice.

Normalised angular integration (NAIN) aims to adjust angular total depth by comparing each segment to the urban average, ensuring that the resulting values are independent of graph size. Likewise, normalised angular choice (NACH) addresses the paradox that segregated configurations often inflate total choice values more than integrated ones. By dividing total choice by total depth, NACH reduces the values of segregated elements, effectively framing choice in cost--benefit terms \citep{Hillier2012a}These normalisations are grounded in a broader principle of space syntax, which seeks to remove the effect of graph size from depth and choice so that systems of different sizes can be compared.

Together, NAIN and NACH allow integration and choice to be compared across radii, systems, and study areas, providing a consistent basis for examining spatial accessibility and segregation in the street network.

NAINr and NACHr were computed at 500 m, 1,000 m, 1,500 m, and 5,000 m. These radii bracket immediate pedestrian reach (\textasciitilde{}500 m), neighbourhood spans (\textasciitilde{}1--1.5 km), and inter-zonal structure (\textasciitilde{}5 km) widely used in angular analyses. Although statistical analyses were restricted to census blocks inside the administrative boundary, the segment map was constructed with a 10-km buffer around the urbanised area to minimise edge effects. Pre-processing in QGIS and DepthmapX included topological simplification; segment centralities were exported and area-weighted to blocks for integration with other layers. 

Decadal thermal layers (Landsat-derived, emissivity-corrected LST calibrated to air temperature, Ta mean) were sampled within 10-m buffers along segments to attach mean and maximum thermal attributes. Thermalised centralities were then defined by penalising least-angle paths traversing hot segments. Using local public-health thresholds, segments were classified into heat-exposure categories: Baseline (Ta mean < 26 $^{\circ}$C), Hazardous (26 $^{\circ}$C $\leq$ Ta mean < 28 $^{\circ}$C), and Extreme (Ta mean $\geq$ 28 $^{\circ}$C) \citep{MaganaVargas2020}

Building on the datasets and preprocessing steps described above, the analysis proceeds in three nested scales---macro, medium, and micro---to move from citywide screening to neighbourhood diagnostics and, finally, street-level simulation of passive cooling strategies. At the macro scale, the dependent variable is the summer daytime air temperature per block, obtained by applying the city-specific transfer function from LST to air temperature. To capture spatial heterogeneity in the relationship between temperature and social proxies of vulnerability, a Geographically Weighted Regression (GWR) is estimated by borough with adaptive bandwidth and a bisquare kernel. 

The selected socio-demographic indicators (e.g., no car, no public health service, economic inactivity, ages 65 or over, ages between 6 to 14, disability, elementary education). The GWR provides, for each block, local slopes and t- statistics which quantify local associations between heat and social conditions; they are not causal effects nor a direct measurement of “vulnerability”.

To translate these local tests into an operational risk layer, the study combines thermal hazard/exposure with statistical evidence of social sensitivity. Thermal hazard is defined by absolute thresholds coherent with local risk communication:

A block exhibits statistical social sensitivity if at least one predictor 𝑘 shows a positive and significant local slope:

The composite thermal--social risk category is then assigned as:

with ancillary “any-social” and “count-social” layers to distinguish broad from specific sensitivity. This macro step yields a conservative hotspot map because it requires both thermal exceedance and local statistical association with at least one social indicator.

At the medium scale, the analysis restricts attention to blocks flagged as hotspots and characterises why they concentrate risk by linking built form and configurational structure. Urban morphology is quantified through Space Matrix indicators and configurational structure is captured with Space Syntax metrics (Integration and Choice- normalized) computed over the Mexico City Street network (administrative boundary) at 500, 1,000, 1,500, and 5,000 m. For each hotspot block, adjacent segments (buffer 15 m) are aggregated as length-weighted means and p90, and the shares of segment length falling into 26$\leq$ Ta <28 and $\geq$ 28∘C are recorded. This coupling reveals whether pedestrian-scale centrality co-occurs with severe thermal environments, and it supports the selection of contiguous, representative clusters (by typology and high accessibility, with emphasis on 500--1,500 m) to be taken to simulation.

At the micro scale, representative clusters with high thermal severity and high pedestrian accessibility are simulated with the Urban Multi-scale Environmental Predictor (UMEP) within QGIS\citep{Lindberg2018} The selection of clusters is based on macro-scale GWR results and fixed air-temperature thresholds.

High-resolution DEM (1.5 m) from INEGI at 1:10,000 scale supplies the terrain and surface models used to build the DEM and DSM, all rasters were resampled to 1 m \citep{INEGI2020}. Building footprints come from Google’s Open Buildings dataset (version 3, extended to Latin America), which provides AI-derived polygons from sub-meter imagery\citep{GoogleResearch2023}. The canopy DSM is derived from multi-year NDVI (2014--2024) computed from Landsat 8/9 Collection-2 Level 2 surface reflectance. From these layers, Sky View Factor (SVF), wall heights and wall aspects are generated to parameterise radiative exchange in street canyons \citep{Lindberg2018} (see table 3 and 4).

The micro scale analysis conducted microclimatic simulations with UMEP/SOLWEIG in three study areas (Case 1: Hospital General; Case 2: Airport; Case 3: Basílica) to estimate the mean radiant temperature (Tmrt) and derive UTCI and PET at three-time steps (10:00, 12:00 and 14:00). SOLWEIG resolves short- and long-wave radiative fluxes in complex urban settings and returns Tmrt, defined as the uniform temperature of an imaginary enclosure that would produce the same net radiative exchange as the real environment on a human body(UNAM Instituto de Geofísica; SMN--CONAGUA). It is a key driver of human energy balance and outdoor thermal load(Lindberg, Holmer and Thorsson, 2008; Lindberg and Grimmond, 2011; Lindberg et al., 2018).

From Tmrt, we derived the two biometeorological indices. The is defined as the air temperature of a reference condition that elicits the same thermo-physiological response (UTCI-Fiala model) as the actual combination of air temperature, wind, humidity and radiation; it is designed as an equivalent temperature scale for outdoor comparability(Bröde, Fiala and others, 2012; Jendritzky, de Dear and Havenith, 2012). The Physiological Equivalent Temperature (PET) is based on the MEMI energy balance model and interprets the outdoor climate as the air temperature of a standard indoor setting producing equivalent physiological responses (core and skin temperature, sweating, etc.) (Matzarakis, Mayer and Iziomon, 1999; Matzarakis, Rutz and Mayer, 2007).

Observation points (POI) were deliberately located along high-integration pedestrian corridors identified with Space Syntax (DepthmapX) using NAIN at two metric radii: r = 500 m and r = 1500 m. These scales were prioritised because r = 500 m captures the local walking scale, useful for identifying everyday streets and junctions, while r = 1500 m represents the neighbourhood-to-district scale, highlighting broader connecting axes. Requiring high integration at both radii allowed us to select “robust spines”---streets important both locally and for through-movement---where mitigation efforts would yield the greatest social return.

Hourly Tmrt fields were then used to compute UTCI and PET for each combination of POI × profile (6) × hour (3). Results were aggregated across zones (hourly means, severity rankings at 14:00) and cross-referenced with SpaceMatrix typologies and thermal--social vulnerability (GWR) at two operative thresholds (hot26 and hot28) to identify where and in which urban forms radiative peaks, highly integrated corridors and socially sensitive groups converged.

