\section{Results: Macro Scale}

The decadal summer climatology reveals a marked thermal gradient across Mexico City, consistent with the well-documented urban heat island (UHI) phenomenon. Average near-surface air temperatures, calibrated from Landsat land surface temperature (LST) using the RedMet network regression, reach approximately 26.8 $^{\circ}$C in central and eastern boroughs such as Cuauhtémoc, Benito Juárez, Iztacalco, Venustiano Carranza and Iztapalapa. In contrast, peripheral areas with abundant vegetation and higher elevations, notably Xochimilco, Milpa Alta and the Ajusco foothills in Tlalpan, register lower summer means of 23--24 $^{\circ}$C. This gradient highlights the amplifying role of dense built form and limited vegetation in the city core, as opposed to the cooling influence of conservation land and montane forests at the edges of the basin.

Beyond absolute temperature levels, derived indices shed light on spatial inequities of thermal stress. The urban heat island intensity (UHI\_air), expressed as the z-score of calibrated air temperature, shows anomalies exceeding +1.5σ in the east and northeast, particularly in Iztapalapa, Gustavo A. Madero and Venustiano Carranza. These boroughs concentrate compact housing, industrial corridors and limited vegetation cover, which coincide with the highest levels of exposure. Spectral indicators confirm these spatial patterns. Blocks with NDVI values below 0.2 and NDBI values above 0.4 consistently correspond to heat hotspots, while slightly higher albedo in low-density western boroughs moderates localized heating. This bio-physical consistency validates the robustness of the GEE-based modelling chain and supports its interpretability in a high-altitude semi-arid environment (see graphic 1 and 2).

Finally, the application of locally relevant thermal thresholds underscores the scale of exposure. Approximately 32 \% of all blocks in Mexico City exceed the 26 $^{\circ}$C health risk threshold, while 11 \% surpass the 28 $^{\circ}$C “extreme heat” level during the summer mean. The spatial concentration of these extremes in the centre-east belt of the city confirms a structural inequality: the densest and most transit-dependent boroughs are also those where residents face the most severe outdoor heat burden (see Fig. 1.1 to1.6).

\begin{figure}[H]
\centering
\includegraphics[width=0.48\textwidth]{../images/social/mapa_Ta_mean.png}
\includegraphics[width=0.48\textwidth]{../images/social/mapa_UHI_mean.png}
\caption{Left: Mean air temperature ($^{\circ}$C) by census block. Right: Urban Heat Island intensity (z-score) showing thermal anomalies across Mexico City.}
\label{fig:thermal_maps}
\end{figure}

\begin{figure}[H]
\centering
\includegraphics[width=0.48\textwidth]{../images/social/mapa_NDVI_mean.png}
\includegraphics[width=0.48\textwidth]{../images/social/mapa_LST_mean.png}
\caption{Left: Mean NDVI by census block showing vegetation distribution. Right: Mean Land Surface Temperature (LST) in $^{\circ}$C.}
\label{fig:biophysical_maps}
\end{figure}

The sociodemographic profile of Mexico City shows that potentially vulnerable groups are not evenly spread across the territory but cluster in specific areas. Children are more common in peripheral boroughs with weaker infrastructure and fewer urban services. Older adults are concentrated in semi-rural and southern boroughs, where health service provision and socioeconomic conditions are generally more limited. Indigenous and Afro-descendant populations, while small in overall proportion, are locally concentrated in marginalised areas with service deficits. People with disabilities, about five percent of the total, are dispersed citywide but face systematic barriers of mobility and accessibility that heighten dependency. Overall, peripheral and semi-rural zones emerge as the main locations where socially vulnerable groups are concentrated, reflecting broader territorial inequalities in access to infrastructure, health services, and urban quality (see Fig. 2.1 to 2.17; see Fig. 3.1 to 3.17).

Before modelling, variables with excessive zero values were excluded to address zero inflation---for instance, pct\_ethnic\_ind (80.15\% zeros), pct\_desocup (77.32\%), pct\_ethnic\_afro (76.03\%), pct\_no\_school (73.51\%) and pct\_0a5 (57.60\%). Multicollinearity was then assessed using variance inflation factors (VIF), which identified problematic indicators such as pct\_serv\_med (VIF = 43.77), pct\_ocup (31.19). These were excluded where necessary. The final candidate set prioritised indicators of mobility, access and demographics, including the proportion of households without a car, individuals without public health coverage, the economically inactive, older adults (65+), children (6--14), persons with disabilities, and those with only primary education (see more information in Appendix E).

Spatial co-location (bivariate LISA) identifies high--high clusters concentrated in eastern and central boroughs, while low--low clusters lie mainly in southern/peri-urban zones. Consistent with these statistics, the central-eastern corridor---covering Iztapalapa, Venustiano Carranza, Gustavo A. Madero and the eastern section of Cuauhtémoc---emerges as a persistent hotspot belt.

Among the covariates, the share of older adults (65+) stands out with a consistently positive association, especially in Milpa Alta, Xochimilco and Tláhuac, where ageing populations coincide with weaker health infrastructure, while effects in central boroughs are weaker or insignificant due to stronger service networks. Children aged 6 to 14 show a smaller and more variable relationship, with significance limited to parts of Iztapalapa and Gustavo A. Madero, pointing to risks mediated by local context. Disability also emerges as relevant, with positive effects in southern and eastern boroughs where accessibility barriers are common, but little impact in central areas. Similarly, car absence has modest overall effects yet becomes significant in peripheral zones where long public transport commutes intensify heat exposure, unlike central boroughs with denser infrastructure. Other variables are more localized: lack of health service affiliation is minor but reinforces vulnerability in the periphery; low educational attainment has near-zero averages yet appears in pockets of the north and east; and economic inactivity trends slightly negative, with few significant results, likely reflecting the mixed composition of inactive groups.

Taken together, these results confirm that socio-demographic effects on Ta\_mean are not uniform but spatially differentiated. Older age and disability exert the most consistent positive associations, while other dimensions such as lack of a car, absence of health coverage, or low schooling levels are more sporadic and context dependent. The eastern and southern boroughs, especially Iztapalapa, Tláhuac, Milpa Alta, Xochimilco and Magdalena Contreras, accumulate the highest and most significant coefficients across multiple variables. By contrast, central boroughs such as Benito Juárez, Cuauhtémoc and Miguel Hidalgo display weaker or non-significant associations. This geography underscores the need for place-based interventions that recognise how social structure and territorial inequalities condition heat exposure across the city (see table 6 and 7).

Across all radii considered, integration (NAIN) values declined steadily as the radius expanded, while choice (NACH) remained relatively stable. This scale-sensitivity highlights that local positional advantage is more variable than route diversity. Crucially, the ranking by hazard category was consistent: hotter areas (hazard 2) always displayed the highest integration and choice, followed by intermediate areas (hazard 1), with cooler areas (hazard 0) at the bottom. At 5 km, NAIN values reached 1.3085 in hazard 2 and 1.1664 in hazard 1, compared with only 0.8186 in hazard 0. Similarly, NACH values stood at 0.9800 in hazard 2 and 0.9362 in hazard 1, both markedly higher than 0.8418 in hazard 0. These results demonstrate that the zones most exposed to heat are also those that are consistently better integrated and offer more route options, underscoring the role of accessibility as a reinforcing factor of thermal vulnerability at this intermediate scale (see Fig. 4, and table 8).

The correlation analysis further substantiates this conclusion (see table 9). Mean air temperature was positively associated with integration, and the strength of this relationship increased with radius---from r = 0.1050 at 500 m to r = 0.2850 at 5 km. Hazard 2 zones thus not only concentrate higher integration values but also sustain this advantage as the scale of analysis widens, suggesting structural configurational benefits that coincide with elevated thermal stress. Hazard 1 areas occupied an intermediate position, maintaining integration values above the city average and clearly differentiated from hazard 0, but below the extremes observed in hazard 2. Vegetation, represented by NDVI, showed the expected negative association with integration, and again the signal was strongest in hazard 2 zones, where built-up form and street network centrality converge with reduced greenness.

This analysis defines hotspot blocks by combining local GWR coefficients and thermal thresholds to ensure that identified areas are simultaneously hot and socially vulnerable. Blocks were classified as hazard when mean air temperature falls between 26--28 $^{\circ}$C and at least one socio-demographic predictor shows a positive and significant local effect (t > 1.96), and as extreme hazard when Ta\_mean $\geq$ 28 $^{\circ}$C with stronger significance (t > 2.58). The resulting maps highlight concentrations of hotspots in eastern and central boroughs, where high temperatures overlap with demographic disadvantages such as low education, high dependency, or lack of mobility assets. By generating both “any-social” and “per-predictor” hotspot layers, the method allows distinguishing whether vulnerability is broad (multiple predictors significant) or specific (e.g., mobility-driven). The value of this approach lies in its strict criteria: only blocks where heat and social vulnerability interact significantly are labeled as hotspots. This provides a robust spatial framework for prioritizing interventions, where extreme hazard areas represent critical targets for adaptation planning and heat-risk mitigation (see Fig. 6.1 to 6.8).
