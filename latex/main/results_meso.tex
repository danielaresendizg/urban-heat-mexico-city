\section{Results: Meso Scale}

The analysis was conducted entirely at the meso scale, examining how both configurational accessibility (via space syntax) and morphological form (via the Space Matrix and typological profiles) converge to shape patterns of urban heat. Although space syntax techniques are often applied at city-wide or metropolitan levels, here they were interpreted as meso-scale indicators of accessibility within and across neighbourhoods, enabling a closer link to morphological conditions and thermal outcomes.

Space Matrix analysis reinforced this picture at the meso scale. The Ground Space Index emerged as the most influential morphological indicator, negatively correlated with NDVI (r = -0.299) and positively associated with Ta\_mean (r = 0.228). Hazard 2 areas typically corresponded to typologies such as Dense Perimeter and Continuous Compact Low, both of which registered high Heat Index values (1.214 and 1.188). These morphologies combine compact built form and limited open space, aligning precisely with the conditions observed in the syntax results: high integration, high thermal exposure, and low vegetation. Hazard 1 areas, while less extreme, included typologies such as Medium Perimeter and Mixed/Undetermined, with Heat Index values around 1.104--1.085, still indicating elevated vulnerability compared to hazard 0 (see table 10-12).

The combined evidence suggests that categories 1 and 2 are not only hotter but also more structurally advantaged in terms of accessibility. Hazard 2 zones retain their positional and configurational advantages across scales, reinforcing thermal risk with mobility and centrality benefits that make them doubly attractive yet environmentally precarious. Hazard 1 zones occupy a transitional position: they share many of the spatial benefits of hazard 2 areas but with somewhat lower heat loads, making them strategic candidates for early mitigation. By contrast, hazard 0 zones are consistently peripheral both thermally and configurationally, their lower integration values aligning with reduced exposure but also lower accessibility.

The meso-scale analysis shows that the city’s hottest areas are also its most integrated ones, and this overlap strengthens as the radius of analysis expands. Morphological indicators confirm that dense and sealed typologies dominate hazard 2, while hazard 1 zones mix compact and transitional forms that still show elevated Heat Index values. The convergence of configurational advantage and thermal vulnerability at this scale provides a powerful rationale for targeted interventions: hazard 2 zones require urgent thermal mitigation through shading, reflective materials, and greening that preserves network legibility, while hazard 1 areas demand a balance of permeability improvements and microclimatic measures to prevent their progression into the highest-risk category.

Finally, a priority zones was developed by integrating morphological typologies with urban accessibility metrics. The procedure began with ranking typologies by thermal severity, based on the proportion of blocks classified as danger (26--<28 $^{\circ}$C) and extreme danger ($\geq$ 28 $^{\circ}$C), and calculating a composite heat index. Within the most exposed typologies, the analysis then selected blocks with the highest levels of configurational accessibility (NAIN/NACH at 500 m and 1,500 m), giving greater weight to the pedestrian scale due to its stronger influence on everyday exposure. These blocks were subsequently aggregated into contiguous zones, applying minimum thresholds of spatial continuity and size, which allowed the definition of robust clusters for analysis at micro scale.

