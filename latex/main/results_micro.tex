\section{Results: Micro Scale}

At the micro scale, we evaluated pedestrian-level heat stress with UMEP/SOLWEIG, forcing the model with zone-specific air temperature and humidity, and computed Tmrt, UTCI and PET at 10:00, 12:00 and 14:00. Observation points were deliberately placed along highly integrated pedestrian corridors (NAIN; radii 500 and 1,500 m) to sample locations where exposure is most likely. We anticipated---and confirm---that, under high-irradiance conditions in Mexico City, Tmrt maxima occur on open, horizontal street-level surfaces (plazas, car parks and wide roads) rather than on rooftops when assessed at pedestrian height. This reflects the city’s latitude (\textasciitilde{}19$^{\circ}$ N), yielding near-zenith solar elevations and short midday shadows, and its high altitude (\textasciitilde{}2,240 m a.s.l.), which reduces optical air mass and often coincides with relatively dry, transparent air---amplifying sun--shade contrast and long-wave emission from heated pavements towards pedestrians (Lindberg et al., 2008; Tian et al., 2023). This contrasts with higher-latitude cities such as London (\textasciitilde{}51$^{\circ}$ N), where lower solar elevations and longer canyon shadows shift peak Tmrt to fully exposed surfaces (e.g., open squares and roof planes), owing to solar geometry and more frequent cloud cover \citep{LindbergGrimmond2011}. Together, these mechanisms explain the spatial patterns observed in our simulations and frame the interpretation of the three study areas (see more information in Appendix F).

Case 1.

In the surroundings of the Hospital General in Mexico City, the results reveal a clear pattern of social vulnerability linked to heat in the 26--28 $^{\circ}$C range. The GWR analysis shows that every block displayed at least one social factor associated with thermal hazard. The most consistent relationships were with low educational attainment (primary education; 16/16 blocks) and the 65+ population (8/16), underscoring these groups as the most sensitive under moderate heat conditions. Other factors, such as children aged 6--14 or lack of medical services, appeared less frequently, while no associations were found under the extreme threshold ($\geq$28 $^{\circ}$C).

These social patterns align closely with the urban fabric of the area, which is dominated by Low Perimeter (37.5\%), Bars (31.2\%) and Medium Perimeter (25.0\%) forms. All typologies fell exclusively within the hot26 regime, with no presence of hot28, and every type exhibited significant GWR associations at this moderate level. In other words, in Zone 1 both the social profile (older adults and low education) and the built form converge in showing vulnerability under moderate but not extreme heat.

The microclimatic simulations reinforce this picture, showing a progressive escalation of heat stress during the day. UTCI rose from 29.6 $^{\circ}$C at 10:00 (moderate stress) to 32.9 $^{\circ}$C at 14:00 (strong stress), while PET reached 37.2 $^{\circ}$C on average by 14:00, with several sites above 41 $^{\circ}$C (very strong). This progression was strongly driven by mean radiant temperature (Tmrt), which climbed from 42.7 $^{\circ}$C to 55.0 $^{\circ}$C, with near-perfect correlations to PET and UTCI. The most severe points, such as POI 7 and POI 2, occurred along highly integrated pedestrian corridors, illustrating how perimeter morphologies embedded in an accessible network exacerbate exposure for socially sensitive groups.

Case 2.

In the area surrounding the Benito Juárez International Airport, vulnerability takes a different form. The GWR results indicate that low educational attainment was the only factor consistently associated with heat, present in 22/24 blocks at 26--28 $^{\circ}$C and in 2/24 at $\geq$28 $^{\circ}$C. Other groups, including older adults and children, showed no significant associations. Here, vulnerability is thus concentrated in educational disadvantage rather than demographic structure.

This is reflected in the SpaceMatrix composition: more than half of the area consists of 0 Property/no cadastre (54.2\%), with a further 29.2\% Low Perimeter and smaller shares of Row/Attached (8.3\%), Mixed (4.2\%) and Open Pavilion (4.2\%). While most forms remain within the hot26 regime, the Low Perimeter and Row/Attached types already show a shift toward hot28 (14.3\% and 50\% of their stock, respectively), with GWR coupling mirroring these higher thresholds. In Zone 2, then, the built form itself marks a transition zone, where typologies begin to express vulnerability under $\geq$28 $^{\circ}$C conditions.

The thermal simulations confirm strong stress throughout the day, with UTCI holding steady at 34--35 $^{\circ}$C and PET fluctuating between 38.7 and 40.8 $^{\circ}$C, with peaks above 41 $^{\circ}$C at 10:00 and 14:00. Tmrt exceeded 63 $^{\circ}$C at these times, explaining the intensity of exposure. The most critical conditions occurred along high-integration corridors, particularly around POI 3, POI 1 and POI 5, where PET surpassed 42 $^{\circ}$C. In this case, educational disadvantage and specific typologies transitioning to hot28 overlap with the busiest, most integrated routes, amplifying the risks for local pedestrians.

Case 3.

In the area surrounding the Basílica de Guadalupe, vulnerability is triggered primarily under extreme heat ($\geq$28 $^{\circ}$C). The GWR results show that 8/9 blocks presented significant associations at this threshold, driven largely by the presence of older adults (65+; 6/9), with isolated links to children (1/9) and low education (1/9). At the lower threshold (26--28 $^{\circ}$C), only 1 block showed any association. Here, therefore, vulnerability emerges mainly when temperatures exceed 28 $^{\circ}$C, with older adults as the principal at-risk group.

This spatial logic is underpinned by the urban morphology, which is dominated by Continuous Compact (44.4\%) and Low Perimeter (33.3\%) forms, with Row/Attached making up the remainder (22.2\%). The compact and perimeter types were entirely situated in hot28, and their social coupling was likewise fully expressed at this level. Row/Attached blocks were divided evenly between hot26 and hot28. Zone 3 thus represents a structural shift into the extreme regime, where both social vulnerability (older adults) and urban form (compact/perimeter blocks) coincide in amplifying exposure.

Microclimatic conditions confirm this severity. UTCI averaged 35--36 $^{\circ}$C across the day, while PET reached 41.5 $^{\circ}$C at 10:00, dipped to 39.5 $^{\circ}$C at 12:00, and returned to 41.6 $^{\circ}$C at 14:00. At both 10:00 and 14:00, most cases (30/36) exceeded 41 $^{\circ}$C, with the maximum at POI 2 (14:00): PET 43.4 $^{\circ}$C, UTCI 37.0 $^{\circ}$C, Tmrt 65.8 $^{\circ}$C. The hottest patches coincided with high-integration corridors, especially POI 2--3--6, reinforcing the convergence of socially sensitive groups, dense morphologies, and busy pedestrian routes.

Taken together, the three cases reveal a progressive gradient of vulnerability across the city. In Case 1 (Hospital General), vulnerability is anchored in moderate heat (26--28 $^{\circ}$C), affecting all typologies equally and strongly associated with older adults and low education. In Case 2 (Airport), moderate stress remains dominant but certain morphologies (Low Perimeter, Row/Attached) already begin to show hot28 signals, again tied to low education. In Case 3 (Basílica), vulnerability emerges almost exclusively at the $\geq$28 $^{\circ}$C threshold, centred on older adults and embedded in compact and perimeter morphologies that structurally intensify exposure.

Across all sites, thermal stress peaks at 14:00 and hotspots consistently align with highly integrated pedestrian corridors, meaning that the very routes of greatest accessibility are also those of highest exposure. The combined evidence highlights a dual risk: social groups with limited adaptive capacity overlap spatially with urban forms and networks prone to extreme heat. This underscores the need for tailored interventions---from continuous shading and cooling refuges on integrated corridors to targeted programmes for older adults and educationally disadvantaged populations---differentiated according to whether zones are operating under moderate or extreme thermal regimes.

